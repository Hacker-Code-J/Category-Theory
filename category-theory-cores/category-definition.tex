\defbox[Category]{\begin{definition}
	A \textbf{category} $\mathcal{C}$ consists of the following components: \begin{itemize}
		\item a class of \textit{objects}, denoted by $\obj{\cC}$; and
		\item a class of \textit{morphisms} (also called \textit{arrows}) from $A$ to $B$, for any objects $A$ and $B$, is denoted by $\text{Hom}_{\mathcal{C}}(A, B)$.
	\end{itemize}
	A category $\cC$ satisfies the following three axioms:
	\begin{enumerate}
		\item (Composition of Morphisms) For any objects $A, B$ and $C$ and any morphisms $f:A\to B$, $g: B\to C$, there exists a \textbf{composite morphism} $g\circ f:A\to C$ in the category $\mathcal{C}$.
		\[\boxed{
		\forall A,B,C\in\obj{\cC}:\forall f\in\hom_{\cC}(A,B):\forall g\in\hom_{\cC}(B,C):\exists g\circ f\in\hom_{\cC}(A,C).}
		\]
		\item (Identity Morphisms) For every object $A \in \obj{\cC}$, there exists the \textbf{identity morphism} $\textsf{id}_A\in\hom_{\cC}(A,A)$ such that for any morphism $f:A\to B$ and $g:B\to A$, $
		\textsf{id}_A \circ f = f$ and $g \circ \textsf{id}_A = g$.
		\[\boxed{
		\forall A, B\in \obj{\cC}: \forall f \in \text{Hom}_{\mathcal{C}}(A, B): \forall g \in \text{Hom}_{\mathcal{C}}(B, A):\text{id}_A \circ f = f\land g \circ \text{id}_A = g}
		\]
		\item (Associativity of Composition) The composition of morphisms must be associative. That is:
		\[\boxed{
		\forall f \in \text{Hom}_{\mathcal{C}}(A, B):\forall g \in \text{Hom}_{\mathcal{C}}(B, C):\forall h \in \text{Hom}_{\mathcal{C}}(C, D): h \circ (g \circ f) = (h \circ g) \circ f
}		\]
	\end{enumerate}
\end{definition}}

\begin{remark}
\ \\\begin{figure}[h!]
	\centering
	\begin{minipage}{.48\textwidth}\centering% https://tikzcd.yichuanshen.de/#N4Igdg9gJgpgziAXAbVABwnAlgFyxMJZABgBpiBdUkANwEMAbAVxiRAEEQBfU9TXfIRQAmclVqMWbAELdeIDNjwEio4ePrNWiEAGFu4mFADm8IqABmAJwgBbJGRA4ISAIzUGWMNpBQ6cAAsjEGpNKR0LOUsbe0R3JxdEURBPbzY-QODQyR9jKJBrOwdqZyRksNyAHUqAYywrGoACSI86ACMYBgAFfmUhECssYwCcAy4gA
		\begin{tikzcd}
			A \arrow[rr, "f", dashed] \arrow[rrdd, "g\circ f"'] &  & B \arrow[dd, "g", dashed] \\
			&  &                           \\
			&  & C                        
		\end{tikzcd}\caption{Composition of Morphisms}
	\end{minipage}%
	\begin{minipage}{.48\textwidth}\centering% https://tikzcd.yichuanshen.de/#N4Igdg9gJgpgziAXAbVABwnAlgFyxMJZABgBpiBdUkANwEMAbAVxiRAEEQBfU9TXfIRQAWclVqMWbAELdxMKAHN4RUADMAThAC2SMiBwQkARmoMsYVohBwI5qCGr1mVkGoC8agDpeAxlg1fAAIfHBgADxw4NWAsKC4AfU5qOAALLDUcJABaACYedS1dRFMDI0R9NIysxDynSVdFd1CIqJi4xPYff0CgxW5eNyK9akMRkAYICDQiYwAOMjVGOBhxBjoAIxgGAAV+PAI2DSxFVKz6lzYWyOjY+KSBwp0TUfLSyemiReXVs03tvbYA5CEDHU7nCSXazXNp3RKyLgULhAA
		\begin{tikzcd}
			A \arrow[rrrr, "f=f\circ \textsf{id}_A", shift left=2] \arrow["\textsf{id}_A"', loop, distance=2em, in=125, out=55] &  &  &  & B \arrow[llll, "g=\textsf{id}_A\circ g", shift left=2] \arrow["\textsf{id}_B"', loop, distance=2em, in=305, out=235]
		\end{tikzcd}\caption{Identity Morphisms}
	\end{minipage}%
\end{figure}
\begin{figure}[h!]\centering\centering% https://tikzcd.yichuanshen.de/#N4Igdg9gJgpgziAXAbVABwnAlgFyxMJZABgBpiBdUkANwEMAbAVxiRAEEQBfU9TXfIRQAmclVqMWbAMLdeIDNjwEiARjHV6zVohAAhOXyWCiAZg0TtbACLdxMKAHN4RUADMAThAC2SMiBwIJFFLKV03QxBPH2DqQKR1UJ0QR0jo30RE+MRzJLYACzSvDP9sxIAjGDAoJAAWAE5NSWTHAB1WgGMsDw6AAgied2K-OKCc6krqpAA2fy0wkHz2rp6ACjbO7r63AEoimMQQ7NzJmsQAWgamq10lzZ7e1OoGOkqGAAV+ZSEQDyxHfI4fYlUZIE5VM7nWbXBarO4rPqOHbLLb9EDUOD5LBuIGIfwvN6fYwqXR-AFArgULhAA
	\begin{tikzcd}
		A \arrow[r, "f"] \arrow[rr, "g\circ f", bend left=49] \arrow[rrr, "h\circ(g\circ f)", bend left=60] \arrow[rrr, "(h\circ g)\circ f"', bend right=60] & B \arrow[r, "g"] \arrow[rr, "h\circ g"', bend right=49] & C \arrow[r, "h"] & D
	\end{tikzcd}\caption{Associativity of Composition}
\end{figure}
\end{remark}

\newpage
\subsection*{Category}
A \textbf{category} \( \mathcal{C} \) consists of the following data:

\begin{itemize}
	\item \textbf{Objects}: A collection of objects, denoted \( \text{Ob}(\mathcal{C}) \).
	\item \textbf{Morphisms}: For each pair of objects \( A, B \in \text{Ob}(\mathcal{C}) \), there is a set \( \text{Hom}_{\mathcal{C}}(A, B) \) of morphisms from \( A \) to \( B \). If \( f \in \text{Hom}_{\mathcal{C}}(A, B) \), we write \( f: A \to B \).
	\item \textbf{Composition}: For any three objects \( A, B, C \in \text{Ob}(\mathcal{C}) \), there is a binary operation
	\[
	\circ: \text{Hom}_{\mathcal{C}}(B, C) \times \text{Hom}_{\mathcal{C}}(A, B) \to \text{Hom}_{\mathcal{C}}(A, C),
	\]
	which assigns to each pair of morphisms \( f: A \to B \) and \( g: B \to C \) a morphism \( g \circ f: A \to C \), called their composition.
	\item \textbf{Identity Morphisms}: For each object \( A \in \text{Ob}(\mathcal{C}) \), there exists a morphism \( \text{id}_A \in \text{Hom}_{\mathcal{C}}(A, A) \), such that for any morphism \( f: A \to B \),
	\[
	\text{id}_B \circ f = f \quad \text{and} \quad f \circ \text{id}_A = f.
	\]
	\item \textbf{Associativity}: For all morphisms \( f: A \to B \), \( g: B \to C \), and \( h: C \to D \), we have
	\[
	h \circ (g \circ f) = (h \circ g) \circ f.
	\]
\end{itemize}

Symbolically, a category \( \mathcal{C} \) is written as:
\[
\mathcal{C} = \left( \text{Ob}(\mathcal{C}), \text{Hom}_{\mathcal{C}}, \circ, \text{id} \right)
\]
where \( \circ \) is the composition operation, and \( \text{id} \) represents the identity morphisms.
\newpage
\subsection*{Functor}
Given two categories \( \mathcal{C} \) and \( \mathcal{D} \), a \textbf{functor} \( F: \mathcal{C} \to \mathcal{D} \) consists of:

\begin{itemize}
	\item \textbf{Object mapping}: For each object \( A \in \text{Ob}(\mathcal{C}) \), there is an object \( F(A) \in \text{Ob}(\mathcal{D}) \).
	\item \textbf{Morphisms mapping}: For each morphism \( f: A \to B \) in \( \mathcal{C} \), there is a morphism \( F(f): F(A) \to F(B) \) in \( \mathcal{D} \).
\end{itemize}

These assignments must satisfy the following properties:

\begin{itemize}
	\item \textbf{Preservation of Identity}: For each object \( A \in \mathcal{C} \),
	\[
	F(\text{id}_A) = \text{id}_{F(A)}.
	\]
	\item \textbf{Preservation of Composition}: For any pair of morphisms \( f: A \to B \) and \( g: B \to C \) in \( \mathcal{C} \),
	\[
	F(g \circ f) = F(g) \circ F(f).
	\]
\end{itemize}

Symbolically, a functor \( F \) from \( \mathcal{C} \) to \( \mathcal{D} \) can be written as:
\[
F: \mathcal{C} \to \mathcal{D}, \quad \left( A \mapsto F(A), \quad f \mapsto F(f) \right)
\]
with the conditions \( F(\text{id}_A) = \text{id}_{F(A)} \) and \( F(g \circ f) = F(g) \circ F(f) \).


\subsection*{Natural Transformation}
Given two functors \( F, G: \mathcal{C} \to \mathcal{D} \), a \textbf{natural transformation} \( \eta: F \Rightarrow G \) is a collection of morphisms \( \eta_A: F(A) \to G(A) \) in \( \mathcal{D} \), one for each object \( A \in \mathcal{C} \), such that for every morphism \( f: A \to B \) in \( \mathcal{C} \), the following diagram commutes:

\[
\begin{array}{ccc}
	F(A) & \xrightarrow{F(f)} & F(B) \\
	\downarrow{\eta_A} &  & \downarrow{\eta_B} \\
	G(A) & \xrightarrow{G(f)} & G(B)
\end{array}
\]

In other words, for every morphism \( f: A \to B \) in \( \mathcal{C} \), the following relation holds in \( \mathcal{D} \):
\[
\eta_B \circ F(f) = G(f) \circ \eta_A.
\]

Symbolically, a natural transformation \( \eta: F \Rightarrow G \) is a family of morphisms \( \{ \eta_A \}_{A \in \mathcal{C}} \), such that for all \( f: A \to B \),
\[
\eta_B \circ F(f) = G(f) \circ \eta_A.
\]


\newpage
\begin{figure}[h!]\centering
	% https://tikzcd.yichuanshen.de/#N4Igdg9gJgpgziAXAbVABwnAlgFyxMJZAJgBoAGAXVJADcBDAGwFcYkQBxAGQH0wAKAEoBKEAF9S6TLnyEUAFgrU6TVu0EA9YAB1teALbwx4ySAzY8BIouLKGLNohABlLboNGTUi7KJlbNPZqTtx8-M6iEt4yVijkSoGqjiCCXmbSlnLI8QEqDuzO4sowUADm8ESgAGYAThD6SPEgOBBIAIyJ+U66MDj0PKlRILX1SADMNC1IZHnBID19PIVDIw2ITVOIEyCM9ABGMIwAChm+TjVYpQAWOCCdc7r69DhXe1XAoWBi-LpoV1iRUyrdqTVqIGZBZKPZ6vd4AVW+v3+gOqdTWimaYIArPcodo-lg7jt9ocTj5YiALtdbmJKGIgA
	\begin{tikzcd}
		R \arrow[dd, "\phi"'] &  & GL_n(R) \arrow[rr, "\eta_R=\det_R"] \arrow[dd, "\mathbf{GL}_n(\phi)"] &  & R^{\times} \arrow[dd, "\mathbf{U}(\phi)"] \\
		&  &                                                                 &  &                                           \\
		S                     &  & GL_n(S) \arrow[rr, "\eta_S=\det_S"]                                    &  & S^{\times}                               
	\end{tikzcd}
\end{figure}
\begin{figure}[h!]\centering
	% https://tikzcd.yichuanshen.de/#N4Igdg9gJgpgziAXAbVABwnAlgFyxMJZABgBpiBdUkANwEMAbAVxiRAB12BbOnACwBGA4AC0AviDGl0mXPkIoyAJiq1GLNpx78hoiVJnY8BIkvKr6zVohABxADIB9JQAotvQcPEBKSdJAYRvKmpCrUlho2Ds5u3B66Pn6GciYoACzm4erWHHE6XmIAepx4XPBJAbLGCsgZYWpWmnmeesXspeViqjBQAOblKKAAZgBOEFxIZCA4EEgAjFmNNpxofFgg1Ax0AjAMAApVwTYjWL18OBWj40hm07OIAMyLkblo2AC87jpDwA5iMSs1r5NttdgcgqkQAwYEMLgYQFcJogMnckABWZ45Ep0JifZoCH4AVTEsVWWGBUNB+0OkOhsMuYyRtxmSBRESx7FgOEcXxa4gZ10e1BZiAxDRenC5PPxBUkFDEQA
	\begin{tikzcd}
		\mathbb{Z} \arrow[dd, "\phi"'] &  & GL_2(\mathbb{Z}) \arrow[dd, "\sigma=\mathbf{GL}_2(\phi)"] \arrow[rr, "\det_\mathbb{Z}"] &  & \mathbb{Z}^\times \arrow[dd, "\tau=\mathbf{U}(\phi)"] \\
		&  &                                                                                       &  &                                                       \\
		\mathbb{Z}                     &  & GL_2(\mathbb{Z}) \arrow[rr, "\det_\mathbb{Z}"]                                        &  & \mathbb{Z}^\times                                    
	\end{tikzcd}
\end{figure}
\begin{figure}[h!]\centering
	% https://tikzcd.yichuanshen.de/#N4Igdg9gJgpgziAXAbVABwnAlgFyxMJZABgBpiBdUkANwEMAbAVxiRAB12BbOnACwBGA4AC0AviDGl0mXPkIoyAJiq1GLNpx78hoiVJnY8BIkvKr6zVohABxADIB9JQAotvQcPEBKSdJAYRvKmpCrUlho2Ds5u3B66Pn6GciYoACzm4erWHHE6XmIAepx4XPBJAbLGCsgZYWpWmnmeesXspeUGlUGpJKQAjBbZbBWBKTVmg1mNNqNVwekDQzMgkqowUADm5SigAGYAThBcSGQgOBBI-dORuWh8WCDUDHQCMAwACvOpIAdYm3wcBVDsckGZzpdEABmG45ThobAAXncOj2wAcYhi8Ievmer3eXx6ChADBgeyBXRBJ0QGQhSAArLCmjg6Exkc0BGiAKpiWL3LC4kn4z7fYmk8nAo7U8EXJC0iJw9iwHCOFEtcSS0HQ6iyxCMhq3TjK1Ucgqa6lnXW0t5gKBygCczywYByUAgOBwGyeBsV2kE3Ik1DgDwliAAtOCXm8RUS2M7sLBzacdZDwQqmn7Oej7JilN6GM7Xe7PXbqDa7dCzlGCaK42AE6xKVKrimkDCSYW2G6PV6yzBbUgw1CzumbGqsxjnPnhYTxnWG0nENc6Xq+wPw2lHSBg1hQ1XOzZuyXp9HZ9V51hE0yxxyA4uAGytxAAdidLq7xd7Poz8TRk7zQYhkCG54qetY2PGl6Nv4VJII+K4ABxvkWPalt+N6ZnegG7sBtLVjGc4QfWUFrGIQA
	\begin{tikzcd}
		\mathbb{Z} \arrow[dd, "\phi"'] \arrow[rrrr, "\mathbf{U}" description, dotted, bend left=49, shift left=2] \arrow[rr, "\mathbf{GL}_2" description, dotted, bend left] &  & GL_2(\mathbb{Z}) \arrow[dd, "\sigma=\mathbf{GL}_2(\phi)"] \arrow[rr, "\det_\mathbb{Z}"] &  & \mathbb{Z}^\times \arrow[dd, "\tau=\mathbf{U}(\phi)"] \\
		{} \arrow[rr, "\mathbf{GL}_2" description, dotted, shift left=4] \arrow[rrrr, "\mathbf{U}" description, dotted, shift right=4]                                       &  & {}                                                                                    &  & {}                                                    \\
		\mathbb{Z} \arrow[rr, "\mathbf{GL}_2" description, dotted, bend right] \arrow[rrrr, "\mathbf{U}" description, dotted, bend right=49]                                 &  & GL_2(\mathbb{Z}) \arrow[rr, "\det_\mathbb{Z}"]                                        &  & \mathbb{Z}^\times                                    
	\end{tikzcd}
\end{figure}
