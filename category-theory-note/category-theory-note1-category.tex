\documentclass[11pt,openany]{article}

\usepackage{mathtools, commath}
% Packages for formatting
\usepackage[margin=1in]{geometry}
\usepackage{fancyhdr}
\usepackage{enumerate}
\usepackage{graphicx}
\usepackage{kotex}
\usepackage{amsmath}
\usepackage{amsthm}
\usepackage[dvipsnames,table]{xcolor}
\usepackage{amssymb, amsfonts}

\usepackage{arydshln} % Include this package
% Fonts
\usepackage[T1]{fontenc}
\usepackage[utf8]{inputenc}
\usepackage{newpxtext,newpxmath}
\usepackage{sectsty}

% Define colors
\definecolor{TealBlue1}{HTML}{0077c2}
\definecolor{TealBlue2}{HTML}{00a5e6}
\definecolor{TealBlue3}{HTML}{b3e0ff}
\definecolor{TealBlue4}{HTML}{00293c}
\definecolor{TealBlue5}{HTML}{e6f7ff}

\definecolor{thmcolor}{RGB}{231, 76, 60}
\definecolor{defcolor}{RGB}{52, 152, 219}
\definecolor{lemcolor}{RGB}{155, 89, 182}
\definecolor{corcolor}{RGB}{46, 204, 113}
\definecolor{procolor}{RGB}{241, 196, 15}

\usepackage{color,soul}
\usepackage{soul}
\newcommand{\mathcolorbox}[2]{\colorbox{#1}{$\displaystyle #2$}}
\usepackage{cancel}
\newcommand\crossout[3][black]{\renewcommand\CancelColor{\color{#1}}\cancelto{#2}{#3}}
\newcommand\ncrossout[2][black]{\renewcommand\CancelColor{\color{#1}}\cancel{#2}}

\usepackage{hyperref}
\usepackage{booktabs}

% Chapter formatting
\definecolor{titleTealBlue}{RGB}{0,53,128}
\usepackage{titlesec}
\titleformat{\section}
{\normalfont\sffamily\Large\bfseries\color{titleTealBlue!100!gray}}{\thesection}{1em}{}
\titleformat{\subsection}
{\normalfont\sffamily\large\bfseries\color{titleTealBlue!50!gray}}{\thesubsection}{1em}{}

%Tcolorbox
\usepackage[most]{tcolorbox}
\usepackage{multirow}
\usepackage{multicol}

%Tikzpicture
\usepackage{tikz}
\usepackage{tikz-cd}
\usetikzlibrary{shapes.geometric, calc}
\usetikzlibrary{positioning}
\usetikzlibrary{angles, quotes}
\usetikzlibrary{patterns,patterns.meta}

\usepackage[linesnumbered,ruled]{algorithm2e}
\usepackage{algpseudocode}
\usepackage{setspace}
\SetKwComment{Comment}{/* }{ */}
\SetKwProg{Fn}{Function}{:}{end}
\SetKw{End}{end}
\SetKw{DownTo}{downto}

% Define a new environment for algorithms without line numbers
\newenvironment{algorithm2}[1][]{
	% Save the current state of the algorithm counter
	\newcounter{tempCounter}
	\setcounter{tempCounter}{\value{algocf}}
	% redefine the algorithm numbering (remove prefix)
	\renewcommand{\thealgocf}{}
	\begin{algorithm}
	}{
	\end{algorithm}
	% Restore the algorithm counter state
	\setcounter{algocf}{\value{tempCounter}}
}

\usepackage{adjustbox}
%Tcolorbox
\usepackage[most]{tcolorbox}
\usepackage{varwidth}
\tcbset{colback=white, arc=5pt}

\newtcolorbox{defbox}[2][]{
	enhanced,
	colframe=defcolor, coltitle=white,
	title={\bf #2},#1
}
\newtcolorbox{thmbox}[2][]{
	enhanced,
	colframe=thmcolor, coltitle=white,
	title={\bf #2},#1
}
% Header and footer formatting
\pagestyle{fancy}
\fancyhead{}
\fancyhf{}
\rhead{\textcolor{TealBlue2}{\textbf{Category-Theory-Note}}}%\rule{3cm}{0.4pt}}
\lhead{\textcolor{TealBlue2}{\textbf{Lecture-Note 1}}}
% Define footer
\newcommand{\footer}[1]{
\begin{flushright}
	\vspace{2em}
%	\includegraphics[width=2cm]{school_logo.jpg} \\
	\vspace{1em}
	\textcolor{TealBlue2}{\small\textbf{#1}}
\end{flushright}
}
%\rfoot{\large Department of Information Security, Cryptogrphy and Mathematics, Kookmin Uni.\includegraphics[height=1.5cm]{school_logo.jpg}}
\fancyfoot{}
\fancyfoot[C]{-\thepage-}

\newcommand{\ie}{\textnormal{i.e.}}
\newcommand{\rsa}{\mathsf{RSA}}
\newcommand{\rsacrt}{\mathsf{RSA}\textendash\mathsf{CRT}}
\newcommand{\inv}[1]{#1^{-1}}

\usepackage{amsthm}
\newtheorem{axiom}{Axiom}[section]
\newtheorem{theorem}{Theorem}
\newtheorem*{theorem*}{Theorem}
\newtheorem{proposition}[theorem]{Proposition}
\newtheorem{corollary}{Corollary}[theorem]
\newtheorem*{corollary*}{Corollary}
\newtheorem{lemma}[theorem]{Lemma}
\newtheorem*{lemma*}{Lemma}

\theoremstyle{definition}
\newtheorem{definition}{Definition}
\newtheorem*{definition*}{Definition}
\newtheorem{remark}{Remark}
\newtheorem{exercise}{Exercise}[section]

%New Command
%\newcommand{\set}[1]{\left\{#1\right\}}
\newcommand{\N}{\mathbb{N}}
\newcommand{\Z}{\mathbb{Z}}
\newcommand{\Q}{\mathbb{Q}}
\newcommand{\R}{\mathbb{R}}
\newcommand{\C}{\mathbb{C}}
\newcommand{\F}{\mathbb{F}}
\newcommand{\nbhd}{\mathcal{N}}
\newcommand{\Log}{\operatorname{Log}}
\newcommand{\Arg}{\operatorname{Arg}}
\newcommand{\pv}{\operatorname{P.V.}}

\newcommand{\of}[1]{\left( #1 \right)} 
%\newcommand{\abs}[1]{\left\lvert #1 \right\rvert}
%\newcommand{\norm}[1]{\left\| #1 \right\|}

\newcommand{\sol}{\textcolor{magenta}{\bf Sol}}
\newcommand{\conjugate}[1]{\overline{#1}}

\newcommand{\res}{\operatorname{res}}
\DeclareMathOperator*{\Res}{\operatorname{Res}}

\renewcommand{\Re}{\operatorname{Re}}
\renewcommand{\Im}{\operatorname{Im}}

\newcommand{\cyclic}[1]{\langle #1 \rangle}
\newcommand{\uniform}{\overset{\$}{\leftarrow}}

\newcommand{\category}{\mathcal{C}}

\newcommand{\obj}[1]{\mathsf{obj}\left(#1\right)}
\newcommand{\homo}[1]{\mathsf{hom}\left(#1\right)}
\newcommand{\mor}[1]{\mathsf{mor}\left(#1\right)}

\newcommand{\id}{\mathsf{id}}

\newcommand{\dom}[1]{\mathsf{Dom}\left(#1\right)}
\newcommand{\cdm}[1]{\mathsf{Cdm}\left(#1\right)}

\newcommand{\op}{\textnormal{op}}


\setstretch{1.25}
\begin{document}
\pagenumbering{arabic}
\begin{center}
	\huge\textbf{Category Theory}\\
	\vspace{0.5em}
	\normalsize{\today}\\
	\vspace{0.5em}
	\large\textbf{Ji, Yong-hyeon}\\
	\vspace{0.5em}
\end{center}

\begin{defbox}{Category}
	A \textbf{category} $\mathcal{C}$ consists of 
	\begin{itemize}
		\item a class $\mathcal{C}_0=\obj{\mathcal{C}}$ of \textbf{objects} $X,Y,Z,\dots$
		\item a class $\mathcal{C}_1=\mathsf{Mor}(\mathcal{C})$ of \textbf{morphisms} (or \textbf{arrow}) $f,g,h,\dots$ between its objects
		\begin{itemize}
			\item for every morphism $f$, an object $\dom{f}$ (called its \textbf{source} or \textbf{domain}), and an object $\cdm{f}$ (called its \textbf{target} or \textbf{codomain})
			\item for every pair of morphisms $f$ and $g$, where $\cdm{f}=\dom{g}$, a morphism $g\circ f$, called \textbf{composite} (also written $gf$ or some times $f;g$)
			\item for every object $X$, a morphism $\mathsf{id}_X$ (or $1_X$), called the \textbf{identity morphism} on $X$;
		\end{itemize}
	\end{itemize}
	The morphisms of $\mathcal{C}$ satisfy the following property:
	\begin{itemize}
		\item[(C1)] (Composition) 
		\begin{align*}
			X,Y,Z\in\obj{\mathcal{C}}\implies &[f\in\morphism{X,Y},g\in\morphism{Y,Z},\cdm{f}=\dom{g}\\
			&\implies \exists(g\circ f)\in\morphism{X,Z}]
		\end{align*}
		\item[(C2)] (Identity)
		\begin{align*}
			X\in\obj{\mathcal{C}}\implies&\exists\mathsf{id}_X\in\morphism{X,X}:\\
			&[Y\in\obj{\mathcal{C}},f\in\morphism{X,Y},g\in\morphism{Y,Z}\\
			&\implies f\circ\mathsf{id}_X=f, \mathsf{id}_X\circ g=g]
		\end{align*}
		\item[(C3)] (Associativity) \[
		f,g,h\in\mathsf{Mor}(\mathcal{C})\implies f\circ(g\circ h)=(f\circ g)\circ h.
		\]
	\end{itemize}
\end{defbox}
\newpage
\begin{remark}
	To describe a metacategory it is necessary to specify:
	\begin{itemize}
		\item The collection $\obj{\mathcal{C}}$
		of objects;
		\item The collection $\mathsf{Mor}(\mathcal{C})$ of morphisms;
		\item For each object $X$, an identity morphism $\mathsf{id}_X:X\to X$;
		\item For every appropriate pair of morphisms $f,g$, the composite morphism $g\circ f$.
	\end{itemize}
	\adjustbox{scale=1.25, center}{
	\begin{tikzcd}
		X \arrow[rr, "f"] \arrow[rrdd, "g\circ f" description] \arrow["\mathsf{id}_X"', loop, distance=2em, in=215, out=145] &  & Y \arrow[dd, "g"] \arrow["\mathsf{id}_Y"', loop, distance=2em, in=35, out=325] \\
		&  &                                                                      \\
		&  & Z \arrow["\mathsf{id}_Z"', loop, distance=2em, in=305, out=235]               
	\end{tikzcd}}
\end{remark}

\footer{Department of Information Security, Cryptology and Mathematics\\
	College of Science and Technology\\
	Kookmin University \\
	Seoul, Republic of Korea}

\newpage
\begin{thebibliography}{99}
	
%\bibitem{schlenga2020cdcl}
%Schlenga, Alexander T. (2020). "Conflict Driven Clause Learning". June 8, 2020.

\bibitem{youtube_intro_to_category_theory}
``Intro to Category Theory'' YouTube, uploaded by Warwick Mathematics Exchange, 1 Feb 2023, \url{https://www.youtube.com/watch?v=AUD2Rpoy6O4}

\bibitem{proofwiki_definition_metacategory}
ProofWiki. ``Definition:Metacategory'' Accessed on [May 05, 2024]. \url{https://proofwiki.org/wiki/Definition:Metacategory}.

\bibitem{nLab_category}
nLab. ``category'' Accessed on [May 05, 2024]. \url{https://ncatlab.org/nlab/show/category#Grothendieck61}.
% Add all the references in the same way
% Use \bibitem for each reference
	
\end{thebibliography}
%\bibliography{category-theory-note-ref}
%\bibliographystyle{abbrv}
\end{document}