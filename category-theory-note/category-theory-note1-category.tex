\documentclass[11pt,openany]{article}

\usepackage{mathtools, commath}
% Packages for formatting
\usepackage[margin=1in]{geometry}
\usepackage{fancyhdr}
\usepackage{enumerate}
\usepackage{graphicx}
\usepackage{kotex}
\usepackage{amsmath}
\usepackage{amsthm}
\usepackage[dvipsnames,table]{xcolor}
\usepackage{amssymb, amsfonts}

\usepackage{arydshln} % Include this package
% Fonts
\usepackage[T1]{fontenc}
\usepackage[utf8]{inputenc}
\usepackage{newpxtext,newpxmath}
\usepackage{sectsty}

% Define colors
\definecolor{TealBlue1}{HTML}{0077c2}
\definecolor{TealBlue2}{HTML}{00a5e6}
\definecolor{TealBlue3}{HTML}{b3e0ff}
\definecolor{TealBlue4}{HTML}{00293c}
\definecolor{TealBlue5}{HTML}{e6f7ff}

\definecolor{thmcolor}{RGB}{231, 76, 60}
\definecolor{defcolor}{RGB}{52, 152, 219}
\definecolor{lemcolor}{RGB}{155, 89, 182}
\definecolor{corcolor}{RGB}{46, 204, 113}
\definecolor{procolor}{RGB}{241, 196, 15}

\usepackage{color,soul}
\usepackage{soul}
\newcommand{\mathcolorbox}[2]{\colorbox{#1}{$\displaystyle #2$}}
\usepackage{cancel}
\newcommand\crossout[3][black]{\renewcommand\CancelColor{\color{#1}}\cancelto{#2}{#3}}
\newcommand\ncrossout[2][black]{\renewcommand\CancelColor{\color{#1}}\cancel{#2}}

\usepackage{hyperref}
\usepackage{booktabs}

% Chapter formatting
\definecolor{titleTealBlue}{RGB}{0,53,128}
\usepackage{titlesec}
\titleformat{\section}
{\normalfont\sffamily\Large\bfseries\color{titleTealBlue!100!gray}}{\thesection}{1em}{}
\titleformat{\subsection}
{\normalfont\sffamily\large\bfseries\color{titleTealBlue!50!gray}}{\thesubsection}{1em}{}

%Tcolorbox
\usepackage[most]{tcolorbox}
\usepackage{multirow}
\usepackage{multicol}

%Tikzpicture
\usepackage{tikz}
\usepackage{tikz-cd}
\usetikzlibrary{shapes.geometric, calc}
\usetikzlibrary{positioning}
\usetikzlibrary{angles, quotes}
\usetikzlibrary{patterns,patterns.meta}

\usepackage[linesnumbered,ruled]{algorithm2e}
\usepackage{algpseudocode}
\usepackage{setspace}
\SetKwComment{Comment}{/* }{ */}
\SetKwProg{Fn}{Function}{:}{end}
\SetKw{End}{end}
\SetKw{DownTo}{downto}

% Define a new environment for algorithms without line numbers
\newenvironment{algorithm2}[1][]{
	% Save the current state of the algorithm counter
	\newcounter{tempCounter}
	\setcounter{tempCounter}{\value{algocf}}
	% redefine the algorithm numbering (remove prefix)
	\renewcommand{\thealgocf}{}
	\begin{algorithm}
	}{
	\end{algorithm}
	% Restore the algorithm counter state
	\setcounter{algocf}{\value{tempCounter}}
}

\usepackage{adjustbox}
%Tcolorbox
\usepackage[most]{tcolorbox}
\usepackage{varwidth}
\tcbset{colback=white, arc=5pt}

\newtcolorbox{defbox}[2][]{
	enhanced,
	colframe=defcolor, coltitle=white,
	title={\bf #2},#1
}
\newtcolorbox{thmbox}[2][]{
	enhanced,
	colframe=thmcolor, coltitle=white,
	title={\bf #2},#1
}
% Header and footer formatting
\pagestyle{fancy}
\fancyhead{}
\fancyhf{}
\rhead{\textcolor{TealBlue2}{\textbf{Category-Theory-Note}}}%\rule{3cm}{0.4pt}}
\lhead{\textcolor{TealBlue2}{\textbf{Lecture-Note 1}}}
% Define footer
\newcommand{\footer}[1]{
\begin{flushright}
	\vspace{2em}
%	\includegraphics[width=2cm]{school_logo.jpg} \\
	\vspace{1em}
	\textcolor{TealBlue2}{\small\textbf{#1}}
\end{flushright}
}
%\rfoot{\large Department of Information Security, Cryptogrphy and Mathematics, Kookmin Uni.\includegraphics[height=1.5cm]{school_logo.jpg}}
\fancyfoot{}
\fancyfoot[C]{-\thepage-}

\usepackage{amsthm}
\newtheorem{axiom}{Axiom}[section]
\newtheorem{theorem}{Theorem}
\newtheorem*{theorem*}{Theorem}
\newtheorem{proposition}[theorem]{Proposition}
\newtheorem{corollary}{Corollary}[theorem]
\newtheorem*{corollary*}{Corollary}
\newtheorem{lemma}[theorem]{Lemma}
\newtheorem*{lemma*}{Lemma}

\theoremstyle{definition}
\newtheorem{definition}{Definition}
\newtheorem*{definition*}{Definition}
\newtheorem{remark}{Remark}
\newtheorem{example}{Example}
\newtheorem{exercise}{Exercise}[section]
\newcommand{\category}{\mathcal{C}}

\newcommand{\obj}[1]{\mathsf{obj}\left(#1\right)}
\newcommand{\homo}[1]{\mathsf{hom}\left(#1\right)}
\newcommand{\mor}[1]{\mathsf{mor}\left(#1\right)}

\newcommand{\id}{\mathsf{id}}

\newcommand{\dom}[1]{\mathsf{Dom}\left(#1\right)}
\newcommand{\cdm}[1]{\mathsf{Cdm}\left(#1\right)}

\newcommand{\op}{\textnormal{op}}


\setstretch{1.25}
\begin{document}
\pagenumbering{arabic}
\begin{center}
	\huge\textbf{Category Theory}\\
	\vspace{0.5em}
	\normalsize{\today}\\
	\vspace{0.5em}
	\large\textbf{Ji, Yong-hyeon}\\
	\vspace{0.5em}
\end{center}

\section{Definition}
\begin{defbox}{Category}
	A \textbf{category} $\mathcal{C}$ consists of 
	\begin{itemize}
		\item A class $\obj{\mathcal{C}}$ of \textbf{objects} $A,B,C,\dots$ in $\category$
		\item For all (ordered) pairs of objects $X,Y\in\obj{\category}$, a class $\homo{A,B}$ of \textbf{maps} or \textbf{arrow} called \textbf{morphisms} from $A$ to $B$, called the \textbf{homo-set} of morphisms from $A$ to $B$. If $f\in\homo{A,B}$, we write $f:A\to B$.
		\begin{itemize}
			\item for every morphism $f$, an object $\dom{f}$ (called its \textbf{source} or \textbf{domain}), and an object $\cdm{f}$ (called its \textbf{target} or \textbf{codomain})
			\item for every pair of morphisms $f$ and $g$, where $\cdm{f}=\dom{g}$, a morphism $g\circ f$, called \textbf{composite} (also written $gf$ or some times $f;g$)
			\item for every object $X$, a morphism $\mathsf{id}_X$ (or $1_X$), called the \textbf{identity morphism} on $X$;
		\end{itemize}
		\item For any three objects $A,B,C\in\obj{\category}$, a binary operation, \[
		\fullfunction{\circ}{\homo{A,B}\times\homo{B,C}}{\homo{A,C}}{(f,g)}{g\circ f},
		\] called \textbf{composition}, such that,
		\begin{itemize}
			\item (\textit{associativity}) \[
			f\in\homo{A,B},g\in\homo{B,C},h\in\homo{C,D}\implies h\circ(g\circ f)=(h\circ g)\circ f;
			\]
			\item (\textit{identity}) \begin{align*}
				X\in\obj{\mathcal{C}}\implies&\exists\mathsf{id}_X\in\homo{X,X}\ \text{such that}\\
				&[f\in\homo{A,X},g\in\homo{X,B}\\
				&\implies \id_X\circ f=f, g\circ\mathsf{id}_X=g]
			\end{align*}
		\end{itemize}
	\end{itemize}
\end{defbox}
\newpage
\begin{remark}
%	\ \\ \adjustbox{scale=1, center}{
%		\begin{tikzcd}
%			A \arrow[rr, "f"] \arrow[rrdd, "g\circ f" description] \arrow["\mathsf{id}_A"', loop, distance=2em, in=215, out=145] &  & B \arrow[dd, "g"] \arrow["\mathsf{id}_B"', loop, distance=2em, in=35, out=325] \\
%			&  &                                                                      \\
%			&  & C \arrow["\mathsf{id}_C"', loop, distance=2em, in=305, out=235]               
%	\end{tikzcd}}
	To describe a acategory it is necessary to specify:
	\begin{itemize}
		\item (Objects) $\obj{\category}=\set{A,B,C,D,E\dots}$\\
		\vspace{8pt}
		\begin{figure}[h!]\centering
			\begin{tikzcd}[remember picture, cells={nodes={draw=none}}]
				| [alias=A] | A &  & B &  & D \\
				&  &   &  &   \\
				&  & C &  & | [alias=E] | E
			\end{tikzcd}
			
			\begin{tikzpicture}[overlay, remember picture]
				% Calculate bounding box coordinates
				\coordinate (TopLeft) at ($(A.north west) + (-0.5,0.5)$);
				\coordinate (BottomRight) at ($(E.south east) + (0.5,-0.5)$);
				% Draw rounded rectangle
				\draw[rounded corners, thick, red] (TopLeft) rectangle (BottomRight);
				% Label the rectangle
				\node at ($(TopLeft)!0.5!(BottomRight) + (0,1.75)$) [above] {$\category$};
			\end{tikzpicture}
		\end{figure}
		
		\item (Morphisms) $\homo{A,B}=\set{f,f',\dots}$; $\homo{A,B}\neq\homo{B,A}$\\
		\vspace{8pt}
		\begin{figure}[h!]\centering
			\begin{tikzcd}[remember picture, cells={nodes={draw=none}}]
				| [alias=A] | A \arrow[rr, "f"] \arrow[rrdd, "h"] \arrow[rr, "f'", bend left=49] \arrow["\mathsf{id}_A"', loop, distance=2em, in=215, out=145] \arrow[rrdd, dashed, bend right=49] \arrow[rrdd, dashed, bend right=49, shift right=3] &  & B \arrow[dd, "g"] \arrow[dd, "g'", bend left=49] \arrow["\mathsf{id}_B"', loop, distance=2em, in=35, out=325] &  & D \arrow[dd, dashed] \arrow[dd, dashed, bend left] \arrow[dd, dashed, bend right] \arrow["\mathsf{id}_D"', loop, distance=2em, in=125, out=55] \\
				&  &                                                                                                               &  &                                                                                                                                                \\
				&  & C \arrow["\mathsf{id}_C"', loop, distance=2em, in=305, out=235]                                               &  & | [alias=E] | E \arrow["\mathsf{id}_E"', loop, distance=2em, in=305, out=235]                                                                               
			\end{tikzcd}
			
			\begin{tikzpicture}[overlay, remember picture]
				% Calculate bounding box coordinates
				\coordinate (TopLeft) at ($(A.north west) + (-1.5,1.5)$);
				\coordinate (BottomRight) at ($(E.south east) + (1.5,-1.5)$);
				% Draw rounded rectangle
				\draw[rounded corners, thick, red] (TopLeft) rectangle (BottomRight);
				% Label the rectangle
				\node at ($(TopLeft)!0.5!(BottomRight) + (0,2.75)$) [above] {$\category$};
			\end{tikzpicture}
		\end{figure}
		\item (Composition) \\ \adjustbox{scale=1, center}{
			\begin{tikzcd}
				&  & B \arrow[rrd, "g"] &  &   \\
				A \arrow[rrrr, "g\circ f"] \arrow[rru, "f"] &  &                    &  & C
		\end{tikzcd}}
		\item (Identity) \begin{center}
			\begin{tikzcd}
				A \arrow[rrrr, "\mathsf{id}_B\circ f=f=f\circ \mathsf{id}_A"] \arrow["\mathsf{id}_A"', loop, distance=2em, in=125, out=55] &  &  &  & B \arrow["\mathsf{id}_B"', loop, distance=2em, in=125, out=55]
			\end{tikzcd}
		\end{center}
		\item (Associativity) \begin{center}
			\begin{tikzcd}
				A \arrow[rr, "f"] \arrow[rrrr, "g\circ f", bend left] \arrow[rrrrrr, "h\circ(g\circ h)" description, bend left, shift left=3] \arrow[rrrrrr, "(h\circ g)\circ f" description, bend right, shift right=3] &  & B \arrow[rr, "g"] \arrow[rrrr, "h\circ g", bend right] &  & C \arrow[rr, "h"] &  & D
			\end{tikzcd}
		\end{center}
	\end{itemize}
\end{remark}

\vfill
\section{Examples}

\begin{example}[Trivial Category]
\ \begin{itemize}
	\item $\obj{\category}=\set{A}$
	\item $\homo{A,A}=\set{\id_A}$
\end{itemize}
\adjustbox{scale=1.25,center}{
\begin{tikzcd}
	A \arrow["\mathsf{id}_A"', loop, distance=2em, in=35, out=325]
\end{tikzcd}}
\end{example}

\begin{example}
	\ \begin{itemize}
		\item $\obj{\category}=\set{A,B}$
		\item $\homo{A,B}=\set{f}$
		\item $\homo{B,A}=\emptyset$
	\end{itemize}
	\adjustbox{scale=1.25,center}{
		\begin{tikzcd}
			A \arrow[r, "f"] & B
	\end{tikzcd}}
\end{example}

\begin{example}
	Let $(G,*)$ be a group. \begin{itemize}
		\item $\obj{\category}=\set{X}$
		\item $\homo{X,X}=\set{G}$
		\item Define $g\circ f:=g*f$
	\end{itemize}
\end{example}

\begin{example}
\ \begin{itemize}
	\item \textbf{Set};\quad $$\text{Set}\xrightarrow[\text{Function}]{}\text{Set}$$
	\item \textbf{Grp};\quad $$\text{Group}\xrightarrow[\text{Homomorphism}]{}\text{Group}$$
	\item \textbf{Top};\quad $$\text{Topological Space}\xrightarrow[\text{Continuous Map}]{}\text{Topological Space}$$
	\item \textbf{Vect}$_K$;\quad $$\text{Vector Space}\xrightarrow[\text{Linear Transformation}]{}\text{Vector Space}$$
\end{itemize}
\end{example}

\newpage
\begin{example}
	\ \begin{itemize}
		\item 
		$f:x\to y$ if and only if $x\leq y$\\
		\adjustbox{scale=1.25, center}{
			\begin{tikzcd}
				x \arrow[rr, "f"]            &  & y \arrow[rr, "g"] &  & z \\
				x \arrow[rrrr, "h"] &  &                   &  & z
		\end{tikzcd}}
		\item $\id_x:x\to x$ if and only if $x\leq x$
	\end{itemize}
	\[
	\begin{array}{c}
		(\R,\leq) \\ \text{Real Number}\xrightarrow[\text{Ordering}]{}\text{Real Number}
	\end{array}
	\]
\end{example}

\vfill
\section{Product and Dual Categories}

\subsection{Product Categories}
\[
\mathcal{C}\times\mathcal{D}
\]
\begin{align*}
\obj{(\mathcal{C}\times\mathcal{D})}&=\obj{\mathcal{C}}\times\obj{\mathcal{D}}\\
\mathsf{hom}_{\mathcal{C}\times\mathcal{D}}((A,B),(A',B'))&=\mathsf{hom}_\category(A,A')\times\mathsf{hom}_{\mathcal{D}}(B,B')
\end{align*}
\[\begin{array}{cc}
	\mathcal{C} &\mathcal{D} \\ A\xrightarrow{f}A' & B\xrightarrow{g}B'
\end{array}\]
\[\begin{array}{c}
	\mathcal{C} \times\mathcal{D} \\ (A,B)\xrightarrow{(f,g)}(A',B')
\end{array}\]

\subsection{Dual Categories}
\[\begin{array}{cc}
	\category & \mathcal{C}^\op \\ A\to B & A\from B
\end{array}\]

\newpage
\section{Functors}
\begin{align*}
	F:\category&\to\mathcal{D}\\
	F:\obj{\category}&\to\obj{\mathcal{D}}\\
	F:\homo{\category}&\to\homo{\mathcal{D}}\\
\end{align*}

\[
\fullfunction{F}{\category}{\mathcal{D}}{A}{F(A)}
\]
\adjustbox{scale=1.25,center}{
\begin{tikzcd}
	A \arrow[rr, "f"]        & {} \arrow[d, maps to] & B    \\
	F(A) \arrow[rr, "F(f)"'] & {}                    & F(B)
\end{tikzcd}}

\newpage
\section{Natural Transformation}
\begin{itemize}
	\item Let \[
	\begin{tikzcd}
		\mathcal{C} \arrow[r, "F", shift left] \arrow[r, "G"', shift right] & \mathcal{D}
	\end{tikzcd}
	\] be categories and functors.
	\item A map 
	% https://tikzcd.yichuanshen.de/#N4Igdg9gJgpgziAXAbVABwnAlgFyxMJZABgBoBGAXVJADcBDAGwFcYkQAdDgW3pwAsAxk2ABhAL4hxpdJlz5CKAEwVqdJq3ZdeA4Y2AARSdNnY8BIuVLE1DFm0QgpMkBjMLLpJbY0On4tRgoAHN4IlAAMwAnCG4kMhAcCCQrEAAjGDAoJAA2BLtNRwAxEBo4fiwInCQAWgBWGkZ6DMYABTlzRRAsMGxYZ0iYuMQEpJSaDKzavJoCvwBxUpByyurEBpAmlvb3C0cevrYTEGjYpBVE5MQAZkaevygIZjTGNhp+GHpsxDBmRkYaDh6FhGOxIGA3ptmjA2h0PPtelh+rNfFoODAgVJKOIgA
	\adjustbox{scale=1.25,center}{\begin{tikzcd}
		& {} \arrow[dd, "\eta" description, Rightarrow] &             \\
		\mathcal{C} \arrow[rr, "F" description, bend left=60, shift left=5] \arrow[rr, "G" description, bend right=60, shift right=5] &                                               & \mathcal{D} \\
		& {}                                            &            
	\end{tikzcd}}
	is a natural transformation
\end{itemize}
% https://tikzcd.yichuanshen.de/#N4Igdg9gJgpgziAXAbVABwnAlgFyxMJZABgBoBGAXVJADcBDAGwFcYkQANEAX1PU1z5CKAEyli1Ok1bsAYgAoOASh58QGbHgJExIyQxZtEIAOKKVvfpqE6K+6UZA9JMKAHN4RUADMAThABbJHIaHAgkMSlDdgAdGJgcegB9LksQP0CkMhAw4JpGLDBHKAgcHFcQGgMZY1lVH38gxGzcxEjqxxNKkAKi9hKyivz6ACMYRgAFAS1hEF8sNwALHHr0xqzQ8MQAZnzC4ohmEcY2GkWYeigkMGZGRlD6LEZ2SD6qh1j4xO7GUfGp6zaYyFbCwZzcIA
\adjustbox{scale=1.25,center}{\begin{tikzcd}
	&  & F(A)\in\obj{\mathcal{D}} \arrow[dd, "\eta_A"] \\
	A\in\obj{\category} \arrow[rru, "F", dotted] \arrow[rrd, "G"', dotted] \arrow[rr, "\eta" description, Rightarrow] &  & {}                        \\
	&  & G(A)\in\obj{\mathcal{D}}                  
\end{tikzcd}}

% https://tikzcd.yichuanshen.de/#N4Igdg9gJgpgziAXAbVABwnAlgFyxMJZABgBpiBdUkANwEMAbAVxiRAEEQBfU9TXfIRRkATFVqMWbAELdeIDNjwEiI8uPrNWiEADEAFOwCUcvksFEALOuqapOgOKGTPMwJUo1Y25O1790i7yiu5CyNbeElpsToGmCvzKYQDMNlH2IPEhSUSpkXZ+3OIwUADm8ESgAGYAThAAtkhkIDgQSACMPtE6VfG1DUhqLW2I1ul+BlVB1XWNiKnDSACsXRlOU32zg9StSAsFbAA6hzA4dAD6nNQMdABGMAwACokWOjVYpQAWOJsDozsjFbjI4nM7nWSuED9OZDXaIIEMLBgPxQOhwT4lEDXO4PZ7mDwgJHYWBY4E6fTHGAADywcBwcAAhEYAPy-ObNOFDRHIthQCA4HCY1YTUk3e5PF4EolYEmQ6EdAFIMYHHS6UVIlH8wVQUU4iX4oSEsDE1hyraIDkjBb3MA6xAANma3M1AqFZJADl14rxoTY0tl8nliE6i3h1w1vK1bptdoAtI7hTEvbjJYb-abA+bLUh7dQY0qAJzhnk6PlMW4MVjY72pv3GmVV92Us6kjF0O1gJgMBg7OhYBhsSA8s1-ENwgDseZgtqQscsRZAzsj5crrZg7aQne7vf7g4IjbFKYNdZNpJVIGbdCKXCAA
\adjustbox{scale=1.25,center}{\begin{tikzcd}
		A \arrow[dd, "f"] \arrow[rr, "F" description, dotted] \arrow[rrrr, "G" description, dotted, bend left=60] \arrow[rrr, "\eta" description, Rightarrow, bend left=49] &  & F(A) \arrow[dd, "F(f)"] \arrow[rr, "\eta_A"'] \arrow[rrdd, "(\exists!)?" description, dashed] & {} & G(A) \arrow[dd, "G(f)"] \\
		&  &                                                                                               &    &                         \\
		B \arrow[rr, "F" description, dotted] \arrow[rrrr, "G" description, dotted, bend right=60] \arrow[rrr, "\eta" description, Rightarrow, bend right=49]               &  & F(B) \arrow[rr, "\eta_B"]                                                                     & {} & G(B)                   
\end{tikzcd}}

\newpage
\footer{Department of Information Security, Cryptology and Mathematics\\
	College of Science and Technology\\
	Kookmin University \\
	Seoul, Republic of Korea}

\newpage
\begin{thebibliography}{99}
	
%\bibitem{schlenga2020cdcl}
%Schlenga, Alexander T. (2020). "Conflict Driven Clause Learning". June 8, 2020.

\bibitem{youtube_intro_to_category_theory}
``Intro to Category Theory'' YouTube, uploaded by Warwick Mathematics Exchange, 1 Feb 2023, \url{https://www.youtube.com/watch?v=AUD2Rpoy6O4}

\bibitem{proofwiki_definition_metacategory}
ProofWiki. ``Definition:Metacategory'' Accessed on [May 05, 2024]. \url{https://proofwiki.org/wiki/Definition:Metacategory}.

\bibitem{nLab_category}
nLab. ``category'' Accessed on [May 05, 2024]. \url{https://ncatlab.org/nlab/show/category#Grothendieck61}.
% Add all the references in the same way
% Use \bibitem for each reference
	
\end{thebibliography}
%\bibliography{category-theory-note-ref}
%\bibliographystyle{abbrv}
\end{document}