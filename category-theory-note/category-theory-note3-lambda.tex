\documentclass[11pt,openany]{article}

\input{category-theory-note-setup}
\input{../category-theory-setup-tcolorbox}
% Header and footer formatting
\pagestyle{fancy}
\fancyhead{}
\fancyhf{}
\rhead{\textcolor{TealBlue2}{\textbf{Category-Theory-Note}}}%\rule{3cm}{0.4pt}}
\lhead{\textcolor{TealBlue2}{\textbf{Lecture-Note 3}}}
% Define footer
\newcommand{\footer}[1]{
\begin{flushright}
	\vspace{2em}
	%	\includegraphics[width=2cm]{school_logo.jpg} \\
	\vspace{1em}
	\textcolor{TealBlue2}{\small\textbf{#1}}
\end{flushright}
}
%\rfoot{\large Department of Information Security, Cryptogrphy and Mathematics, Kookmin Uni.\includegraphics[height=1.5cm]{school_logo.jpg}}
\fancyfoot{}
\fancyfoot[C]{-\thepage-}

\usepackage{amsthm}
\newtheorem{axiom}{Axiom}[section]
\newtheorem{theorem}{Theorem}
\newtheorem*{theorem*}{Theorem}
\newtheorem{proposition}[theorem]{Proposition}
\newtheorem{corollary}{Corollary}[theorem]
\newtheorem*{corollary*}{Corollary}
\newtheorem{lemma}[theorem]{Lemma}
\newtheorem*{lemma*}{Lemma}

\theoremstyle{definition}
\newtheorem{definition}{Definition}
\newtheorem*{definition*}{Definition}
\newtheorem{remark}{Remark}
\newtheorem{example}{Example}
\newtheorem{exercise}{Exercise}[section]
\newcommand{\category}{\mathcal{C}}

\newcommand{\obj}[1]{\mathsf{obj}\left(#1\right)}
\newcommand{\homo}[1]{\mathsf{hom}\left(#1\right)}
\newcommand{\mor}[1]{\mathsf{mor}\left(#1\right)}

\newcommand{\id}{\mathsf{id}}

\newcommand{\dom}[1]{\mathsf{Dom}\left(#1\right)}
\newcommand{\cdm}[1]{\mathsf{Cdm}\left(#1\right)}

\newcommand{\op}{\textnormal{op}}


\setstretch{1.25}
\begin{document}
\pagenumbering{arabic}
\begin{center}
	\huge\textbf{Lambda Calculus}\\
	\vspace{0.5em}
	\normalsize{\today}\\
	\vspace{0.5em}
	\large\textbf{Ji, Yong-hyeon}\\
	\vspace{0.5em}
\end{center}

$\lambda\src.\target$
\begin{lstlisting}
function (input) {
	return output
}
\end{lstlisting}

$\lambda a.\lambda b.$
\begin{lstlisting}
def add (a, b):
	return a + b
	
add(5,7)

def add (a):
	def adda(b):
		return a + b
	return adda

add(5)(7)
\end{lstlisting}
$\lambda a. (\lambda b.(a+b))(5)(7)$

$\true=\lambda a.\lambda b. a$\\
$\false=\lambda a.\lambda b. b$

$\lambda\texttt{bool}.\texttt{bool}(t)(f)$\\
- returned if bool is true\\
- returned if bool is false

\paragraph*{Logic Gates}
\begin{itemize}
	\item $\lnot = \lambda b. b(\false)(\true)$
	\begin{itemize}
		\item returned if $b$ is true
		\item returned if $b$ is false
	\end{itemize}
	\item $\lor = \lambda a.\lambda b. a(\true)(b)$
	\begin{itemize}
	\item returned if $a$ is true
	\item returned if $b$ is false
	\end{itemize}
	\item $\land = \lambda a. \lambda b. a(b)(\false)$
	\begin{itemize}
	\item returned if $a$ is true
	\item returned if $b$ is false
	\end{itemize}
\end{itemize}
%\begin{tikzpicture}[sibling distance=10em,
%%	every node/.style = {
%%		shape=rectangle, rounded corners,
%%		draw, align=center
%%		top color=white, bottom color=blue!20
%%		}
%	level 1/.style={sibling distance=15em},
%	level 2/.style={sibling distance=6em}
%	]
%	\node {Grandparents}
%	child { node {Parent1}
%		child { node {Child1} }
%		child { node {Child2} } }
%	child { node {Parent2}
%		child { node {Child3} }
%		child { node {Child4} } }
%	child { node {Parent3}
%		child { node {Child5} }
%		child { node {Child6} }
%		child { node {Child7} } };
%\end{tikzpicture}

\begin{tikzpicture}[sibling distance=10em,
	%	every node/.style = {
		%		shape=rectangle, rounded corners,
		%		draw, align=center
		%		top color=white, bottom color=blue!20
		%		}
	level 1/.style={sibling distance=15em},
	level 2/.style={sibling distance=6em}
	]
	\node {$\lambda$}
	child { node {...}}
	child { node {$\lambda$}
		child { node {...} }
		child { node {$\lambda$} 
			child { node {...} }
			child { node {$\lambda$...} }
		} };
\end{tikzpicture}

\begin{align*}
	1&=\lambda f.\lambda a.f(a)\\
	2&=\lambda f.\lambda a.f(f(a))\\
	3&=\lambda f.\lambda a.f(f(f(a)))\\
	4&=\lambda f.\lambda a.f(f(f(f(a))))\\
\end{align*}

\[
3(f)(a)=f(f(f(a)))
\]

\begin{align*}
	+1 &= \lambda n.\lambda f.\lambda a. f(n(f)(a))\\
	+ &= \lambda x.\lambda y.x(+1)(y)\\
	* &=\lambda x.\lambda y.y(+(x))(0) \\
	** &=\lambda x.\lambda y.y(*(x))(1)
\end{align*}

\newpage
\section*{Introduction to Axiomatic Thinking}

\[
2+3*7
\]
\[
2+7+7+7
\]
\begin{itemize}
	\item Multiplication is defined as multiple addition
	\item Some rules are defined in terms of other rules
	\begin{itemize}
		\item Multiplication is ``redundant''
	\end{itemize}
\end{itemize}


\newpage
\footer{Department of Information Security, Cryptology and Mathematics\\
	College of Science and Technology\\
	Kookmin University \\
	Seoul, Republic of Korea}

\newpage
\begin{thebibliography}{99}
	
	%\bibitem{schlenga2020cdcl}
	%Schlenga, Alexander T. (2020). "Conflict Driven Clause Learning". June 8, 2020.
	
	\bibitem{youtube_a_byte_of_code}
	``Why functions are turing complete (Lambda Calculus)'' YouTube, uploaded by A Byte of Code, 4 Sep 2022, \url{https://www.youtube.com/watch?v=m32kbFXBRR0}
	
	\bibitem{youtube_advait_shinde}
	``Lambda Calculus vs. Turing Machines (Theory of Computation)'' YouTube, uploaded by Advait Shinde, 3 Mar 2020, \url{https://www.youtube.com/watch?v=ruOnPmI_40g}
	
\end{thebibliography}
%\bibliography{category-theory-note-ref}
%\bibliographystyle{abbrv}
\end{document}