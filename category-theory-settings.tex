\usepackage[utf8]{inputenc}

% =========================================
% Colors
\usepackage[dvipsnames,table]{xcolor}
\definecolor{titleblue}{RGB}{0,53,128}
\definecolor{chaptergray}{RGB}{140,140,140}
\definecolor{sectiongray}{RGB}{180,180,180}

\definecolor{thmcolor}{RGB}{231, 76, 60}
\definecolor{defcolor}{RGB}{52, 152, 219}
\definecolor{lemcolor}{RGB}{155, 89, 182}
\definecolor{corcolor}{RGB}{46, 204, 113}
\definecolor{procolor}{RGB}{241, 196, 15}
\definecolor{execolor}{RGB}{90, 128, 127}
% =========================================

\usepackage[backend=biber, style=numeric]{biblatex}

% Preamble with packages, custom commands, etc.
\usepackage{mathtools}
\usepackage{commath, amsmath, amsthm}
\usepackage{polynomial}
\usepackage{enumerate}
\usepackage{soul} % highlight
\usepackage{lipsum}  % Just for generating text

% Hyperlinks
\usepackage{hyperref}
\hypersetup{
	colorlinks=true,
	linkcolor=titleblue,
	filecolor=black,      
	urlcolor=titleblue,
}

% Tikz
\usepackage{tikz}
\usepackage{tikz-cd}
\usepackage{pgfplots}
\usetikzlibrary{trees}

% Theorem
\newtheorem{axiom}{Axiom}[chapter]
\newtheorem{theorem}{Theorem}[chapter]
\newtheorem{proposition}[theorem]{Proposition}
\newtheorem{corollary}{Corollary}[theorem]
\newtheorem{lemma}[theorem]{Lemma}

\theoremstyle{definition}
\newtheorem{definition}{Definition}[chapter]
\newtheorem{remark}{Remark}[chapter]
\newtheorem{exercise}{Exercise}[chapter]
\newtheorem{example}{Example}[chapter]
\newtheorem*{note}{Note}

% =========================================
% Fonts
\usepackage[T1]{fontenc}
\usepackage[utf8]{inputenc}
\usepackage{newpxtext,newpxmath}
\usepackage{sectsty}
\allsectionsfont{\sffamily\color{titleblue}\mdseries}

% Page layout
\usepackage{geometry}
\geometry{a4paper,left=.8in,right=.6in,top=.75in,bottom=1in,heightrounded}
\usepackage{fancyhdr}
\fancyhf{}
\fancyhead[LE,RO]{\thepage}
\fancyhead[LO]{\nouppercase{\rightmark}}
\fancyhead[RE]{\nouppercase{\leftmark}}
\renewcommand{\headrulewidth}{0.5pt}
\renewcommand{\footrulewidth}{0pt}

% Chapter formatting
\usepackage{titlesec}
\titleformat{\chapter}[display]
{\normalfont\sffamily\Huge\bfseries\color{titleblue}}{\chaptertitlename\ \thechapter}{20pt}{\Huge}
\titleformat{\section}
{\normalfont\sffamily\Large\bfseries\color{titleblue!100!gray}}{\thesection}{1em}{}
\titleformat{\subsection}
{\normalfont\sffamily\large\bfseries\color{titleblue!75!gray}}{\thesubsection}{1em}{}

% Table of contents formatting
\usepackage{setspace}
\usepackage{booktabs}
\usepackage{tocloft}
\renewcommand{\cftchapfont}{\sffamily\color{titleblue}\bfseries}
\renewcommand{\cftsecfont}{\sffamily\color{chaptergray}}
\renewcommand{\cftsubsecfont}{\sffamily\color{sectiongray}}
\renewcommand{\cftchapleader}{\cftdotfill{\cftdotsep}}
% =========================================

% Custom commands
\newcommand{\seminarName}[1]{\textbf{#1}}

\newcommand{\ie}{\textnormal{i.e.}}

\newcommand{\N}{\mathbb{N}}
\newcommand{\Z}{\mathbb{Z}}
\newcommand{\Q}{\mathbb{Q}}
\newcommand{\R}{\mathbb{R}}
\newcommand{\C}{\mathbb{C}}
\newcommand{\F}{\mathbb{F}}


\newcommand{\B}{\mathbb{B}}

\newcommand{\true}{\mathsf{T}}
\newcommand{\false}{\mathsf{F}}

